% !TeX program = xelatex
\title{Matthew Barber}
\author{Matthew Barber}
\date{\today}

\documentclass[11pt]{article}

% prevent hyphen breaks
\tolerance=1
\emergencystretch=\maxdimen
\hyphenpenalty=10000
\hbadness=10000

\usepackage[hidelinks]{hyperref}

\usepackage{graphicx}

\usepackage{geometry}
\geometry{
	a4paper,
	total={120mm,257mm},
	left=20mm,
	top=20mm,
	% 50mm to play
	marginparsep=10mm,
	marginparwidth=50mm % overflow
}

\setlength{\parindent}{0in}

\usepackage{color}
\usepackage{xcolor}

% Material Design colors:
\colorlet{primary}{black!87!white}   % - high emphasis
\colorlet{secondary}{black!60!white} % - med emphasis
\colorlet{tertiary}{black!38!white}  % - disabled

\usepackage{titlesec}
\usepackage{titling}
\usepackage{fontspec}

\newfontfamily\titlefont[Color=primary, Scale=3.75]{Merriweather}
\setmainfont{Roboto}
\newfontfamily\sectionfont[Color=primary]{Roboto Condensed}
\newfontfamily\sectionfontlite[Color=primary]{Roboto Condensed-Light}
\newfontfamily\subsectionfont[Color=primary]{Roboto Slab-Bold}
\titleformat*{\section}{\LARGE\sectionfontlite}
\titleformat*{\subsection}{\large\subsectionfont}
\newcommand{\datedsubsection}[2]{{\large\subsectionfont{#1}} \hfill {\scriptsize\textit{#2}}}
\titlespacing*{\section}{0pt}{6pt}{0pt}
\titlespacing*{\subsection}{0pt}{3pt}{0pt}


\newfontfamily{\annfont}[Color=secondary]{Roboto}
\newfontfamily\marginemph[Color=secondary]{Roboto Condensed}
\newcommand{\annotate}[1]{
	\marginpar{\small{\annfont#1}}
}

\newcommand{\icon}[1]{
	\raisebox{-3pt}{\def\svgwidth{12pt}\input{#1.pdf_tex}}
}

\usepackage{enumitem}
\newcommand{\listhighlights}[1]{
	%\smallskip{}
	\begin{itemize}[topsep=2pt, partopsep=0pt, itemsep=2pt, parsep=0pt, leftmargin=12pt]
		#1
	\end{itemize}
}
\newcommand{\listskills}[2]{
	\textbf{\marginemph #1}
	\begin{description}[topsep=0pt, noitemsep, labelsep=0pt]
	#2
	\end{description}
}

\usepackage{tabularx}

\usepackage{fontawesome}

\begin{document}
\thispagestyle{empty} % removes page number
\raggedright
\color{primary}

{\titlefont Matthew Barber}

\begin{tabularx}{\textwidth}{ @{} l c r @{} }
	\faGlobe{} Southend-on-Sea, Essex, UK &
	%\faPhone{} 07951 415676 &
	\faEnvelope{} \href{mailto:quitesimplymatt@gmail.com}{quitesimplymatt@gmail.com}
\end{tabularx}

\section*{EXPERIENCE}

\annotate{
	\textbf{\marginemph FIND ME ONLINE}
	\begin{description}[labelsep=1pt]
		\item \icon{github} \href{https://github.com/honno/}{github.com/honno}
		\item \icon{jekyll} \href{https://blog.honno.dev/}{blog.honno.dev}
		\item \icon{linkedin} \href{https://www.linkedin.com/in/honno/}{linkedin.com/in/honno}
	\end{description}

	\smallskip

	\listskills{OSS CONTRIBUTIONS}{
		\item NumPy
		\item PyTorch
		\item CuPy
		\item JAX
		\item pandas
		\item PyArrow
		\item polars
		\item Vaex
		\item cuDF
		\item modin
		\item OpenBLAS
		\item Hypothesis
		\item pandera
		\item Rich
	}

	\medskip

	\listskills{LANGUAGES}{
		\item Python
		\item Java
		\item C\#
		\item SQL
		\item JavaScript
		\item Bash
		\item Lisp
	}

	\medskip

	\listskills{TOOLS}{
		\item pytest
		\item git
		\item GitHub Actions
		\item Jupyter Notebook
		\item Linux
	}
}

\datedsubsection{Software Engineer @ \href{https://www.quansight.com/}{Quansight}}{July 2021—}

\listhighlights{
	\item Helped design API standards as a member of the Python Data APIs consortium, which are now adopted by popular array and dataframe libraries such as NumPy and pandas.
	\item Built a REST API backend for a client's image detection platform, using FastAPI and various other libraries that modelled, queried and migrated a postgres database.
	\item Transitioned a client’s ETL pipeline to the Dagster data orchestration
	platform for Python, in which I conducted a refactor of their codebase and
	made deep structural improvements.
	\item Responsible for instilling a testing culture in a client's data pipelines team. I encouraged personnel to write tests and CI infrastructure by writing guides, running tutorials and pair programming.
	\item Contributed NumPy compatibility layers to PyTorch as part of a contract with Meta. Such layers enable scientific code originally built for NumPy to now take advantage of PyTorch tensors and ops.
}

\section*{PROJECTS}

\subsection*{\href{https://github.com/HypothesisWorks/hypothesis/}{Hypothesis}}

Popular property-based testing library for Python that I help maintain.

\listhighlights{
	\item Contributed array API tools that are now used in the test suites of libraries such as NumPy and PyTorch.
	\item Maintained modules used to test numerical/scientific code generally.
}

\subsection*{\href{https://github.com/data-apis/dataframe-interchange-tests/}{dataframe-interchange-tests} \& \href{https://github.com/data-apis/array-api-tests/}{array-api-tests}}

Compliance test suites for the dataframe interchange protocol and array API standards respectively.

\listhighlights{
	\item Employed property-based testing to generate many varied and interesting test cases to say with a high degree of confidence whether an implemented API is compliant or not.
	\item Architected library-agnostic tests so that the test suites can run against any implementation of a standard.
	\item Projects like NumPy and PyArrow have found dozens of bugs in their main libraries just by testing their API compatibility layers.
}

\section*{PUBLICATIONS}

\href{https://conference.scipy.org/proceedings/scipy2023/aaron_meurer.html}{Consortium for Data API Standards, 'Python Array API Standard: Toward Array Interoperability in the Scientific Python Ecosystem', in \textit{Proceedings of the 22nd Python in Science Conference}, 2023, pp. 8–17.}

\section*{EDUCATION}

\datedsubsection{Aston University}{Sep. 2016—July 2020}

1st (Honours) BSc Computer Science

\end{document}
